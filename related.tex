\section{Related Work}
\subsection{Rust-based system}
\begin{itemize}
    \item Theseus
    \item Redleaf
    \item https://sigops.org/s/conferences/hotos/2023/papers/burtsev.pdf
    \item Michael Swift works
\end{itemize}

%\subsection{Other eBPF safety}
\subsection{Unprivileged eBPF through Extra Safety}
Other work has attempted to add additional safety to eBPF programs largely
    to support unprivileged eBPF.
MOAT~\cite{lu2023moat} uses hardware based Intel Memory Protection Keys 
    to isolate eBPF programs from the kernel.
This would prevent attackers from using eBPF.
Another system SandBPF~\cite{sandbpf} uses software fault isolation to dynamically
    sandbox eBPF programs.

\subsection{Other OS extension}
A large body of work exists around the theme of creating extensible operating systems.
Systems such as SPIN \cite{spin} utilize the type safety of Modula-3 to produce safe extensions that dynamically link into the kernel.
VINO \cite{vino} uses software fault isolation techniques and interposition to allow for kernel functions to be replaced, or event-based extensions to be implemented.
Similarly, SLIC \cite{slic} uses interposition as a mechanism to allow trusted extensions to be run with only minimal operating system changes.
