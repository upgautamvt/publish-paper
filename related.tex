\section{Related Work}

% eBPF
%   eBPF for X papers
%   safety problems
%     work on eBPF safety
% We try to rethink about the problem
\para{Linux eBPF} is an active project with lots of applications
~\cite{BMC,Electrode,DINT,Hoiland-Jorgensen:conext:2018,Zhong:osdi:2022,jia2023,ebpf-kcfi}.
% researched in past literatures.
%Works on distributed systems and
%    networking~\cite{BMC,Electrode,DINT,Hoiland-Jorgensen:conext:2018} has
%    exploited eBPF's ability to safely bypass the expensive kernel networking
%    stack without kernel modifications and thereby optimizing
%    system performance.
%At the same time, eBPF has also seen its application in file and storage
%    systems~\cite{Zhong:osdi:2022}, bringing both efficiency and flexibility.
%Other works has explored the use case of eBPF in
%    security~\cite{jia2023,ebpf-kcfi}, where eBPF is used for program system
%    call filters and kernel control flow integrity policies.
In recent years, as safety and security issues of the eBPF infrastructure begin
    to
    surface~\cite{EPF,CVE-2021-31440,CVE-2022-23222,CVE-2022-2785,
        CVE-2021-29154,CVE-2021-45402},
    more works begin to propose new ways to ensure safety and security in eBPF
    via fuzzing~\cite{hung2023brf}, formal
    verification~\cite{ebpf-jit-formal,proof-carrying-verifier},
    dynamic sandboxing~\cite{sandbpf}, and hardware protection
    mechanisms~\cite{lu2023moat,ebpf-pks}.
\projname{} identifies the fundamental \gap{} existing in the current eBPF design and
    proposes a rethink of the problem using a safe-language to solve this challenge.
%It employs a safe language for extension programming to try to solve this
%    fundamental challenge.

% Safety and usability of kernel extensions
%   SPIN and VINO
%   difference in philosophical design
%   modern context
\para{Safe kernel extensions}
A large body of work exists around the theme of creating safe extensible operating
    systems.
SPIN \cite{spin} utilizes the type safety of Modula-3 to produce
    safe extensions that dynamically link into the kernel.
VINO \cite{vino} uses software fault isolation techniques and interposition to
    allow for kernel functions to be replaced, or event-based extensions to be
    implemented.
Similarly, SLIC \cite{slic} uses interposition as a mechanism to allow trusted
    extensions to be run with only minimal operating system changes.
Nooks~\cite{nooks} followed in the same vein while KSplit~\cite{ksplit} attempted to
automate the process of isolating kernel drivers by identifying and synchronizing
shared kernel and driver state using static analysis.
% \jinghao{I have some trouble articulating the difference in philosophical design}
% \mvle{my attempt}
\jinghao{It seems that nooks and ksplit does not belong here.}
We leverage and extend these bodies of work by utilizing a safe language combined with
a thin run time to achieve a safe and efficient kernel extension framework.

% Rust-base safety
%   Rust-based OS
%   Linux kernel Rust support
%   not for kernel extensions
\para{Safe languages in OS kernels}
Past works have also leveraged the safety of Rust or other safe languages in
    operating system design.
Microsoft has developed Singularity~\cite{singularity}, an OS based on the safe
    language Sing\#.
Redleaf~\cite{redleaf} and Theseus~\cite{theseus,theseus-kisv23} both utilizes
    Rust as the kernel programming language, demonstrating the safety and
    isolation benefits stemming from an intra-language design.
Burtsev, et al.~\cite{Burtsev-hotos23}, have proposed the use of Rust to secure
    kernel-driver interfaces.
The use of Rust is also embraced by modern, mainstream operating system
    kernels, with both Linux~\cite{rust-for-linux-doc} and
    Windows~\cite{rust-for-windows} incroporting it into their codebase.
While \projname{} also applies Rust to the kernel context, its goals is to use
    Rust to create safe and usable kernel extensions and has to work with
    unsafe kernel code, which differs from other works that explore an OS
    written in safe language.

% \subsection{Rust-based system}
% \begin{itemize}
%     \item Theseus
%     \item Redleaf
%     \item Evolving Operating Systems Towards Secure Kernel-Driver Interfaces
%     \item Michael Swift works
%     \item Leveraging Rust for Lightweight OS Correctness
%     \item Cross-Language Attacks from NDSS 2022
% \end{itemize}

%\subsection{Other eBPF safety}
% \subsection{Unprivileged eBPF through Extra Safety}
% Other work has attempted to add additional safety to eBPF programs largely
%     to support unprivileged eBPF.
% MOAT~\cite{lu2023moat} uses hardware based Intel Memory Protection Keys
%     to isolate eBPF programs from the kernel.
% This would prevent attackers from using eBPF.
% Another system SandBPF~\cite{sandbpf} uses software fault isolation to dynamically
%     sandbox eBPF programs.
