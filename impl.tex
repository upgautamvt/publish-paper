\section{Implementation}
\label{sec:impl}
%We now discuss how of \projname{} is implemented.
The implementation of \projname{} consists of three parts, the \projname{}
    kernel crate, the kernel support, and the compiler support.

\subsection{\projname{} kernel crate}
\label{impl:kernel-crate}
The kernel crate uses a mixture of safe and unsafe Rust to implement the safe
    interface for \projname{} programs to interact with the kernel.
This interface contains the helper function interface, the kernel data type
    bindings, and the wrapper type for program argument.
%     \milo{Missing portion?}

\para{Helper function interface:}
\projname{} implements its helper interface on top of the kernel helpers, with
    wrapping code that allows programs to invoke helpers with safe Rust types.
Doing so comes at a cost of losing the load time optimization (e.g. inlining) on
    certain helpers by the eBPF verfier, which we further evaluate in
    \S\ref{eval:inline}.

\para{Kernel data type bindings} are generated for the program to access kernel
    data types defined in C.
\projname{} uses \emph{rust-bindgen}~\cite{bindgen} to automatically generate
    kernel bindings and integrates it into the build process of the program.

\para{Wrapping of program argument:}
eBPF hides irrelevant data fields of the kernel internal struct that serves as
    the program argument.
This done by providing the program a "user" struct with only needed fields and
    rewrite accessing instructions at verification time to the kernel struct.
\projname{} implements this by encapsulating the generated binding of the
    kernel internal struct in a Rust struct, in which each field accessible by
    the extension program has an associated getter method.

\subsection{Kernel support}
\label{impl:kernel}
\projname{} implements its program load code and the runtime environment in the
    kernel.
To load the extension program, \projname{} extends the \texttt{bpf} system
    call to able to handle a \projname{} program.
The system call parses the ELF executable and locate all the \texttt{LOAD}
    segments in the executable.
It then allocates new pages and maps the \texttt{LOAD} segments into the kernel
    address space based on the size and permissions of the segments.
The system is also responsible for fixups on the program code to resolve
    referenced kernel helpers and eBPF maps.
The \projname{} runtime environment in the kernel consists of the stack
    unwinding mechanisms (\S\ref{principle:eh}), the support for dedicated
    kernel stack (\S\ref{principle:stack}), and the support for program
    termination (\S\ref{principle:termination}).

\subsection{Compiler support}
\label{impl:compiler}
\projname{} employs a compiler pass to handle the \projname{}-specific
    compile time instrumentations on the stack (\S\ref{principle:stack}).
We take advantage of Rust's use of LLVM~\cite{llvm} as its default code
    generation backend and implement the needed pass in LLVM.
A \projname{}-specific compiler switch is also added to the Rust compiler
    frontend (\emph{rustc}~\cite{rustc}) to gate the \projname{} compiler pass.
