\section{Design and Implementation}

\subsection{Design goal}
\begin{itemize}
    \item Safety
        \begin{itemize}
            \item Same level of safety as eBPF
            \item Memory safety
            \item Control-flow soundness
            \item Resource management
            \item Program termination
        \end{itemize}
    \item Expressiveness
        \begin{itemize}
            \item Support more complicated/advanced programs
            \item Longer programs
            \item Unbounded loops
        \end{itemize}
    \item An important note: we want expressiveness w/o impairing safety (e.g.
        allow unbounded loop while ensuring termination)
\end{itemize}

\subsection{Overview}
\begin{itemize}
    \item Brings out the Rust based approach (use Jiyuan's property-oriented
        argument: We want these properties, and Rust happens to provide these)
    \item Infrastructure (Need a figure similar to Fig. 5 from HotOS paper)
\end{itemize}

\subsection{Runtime crate as the programming interface}
\begin{itemize}
    \item program type (how Rust type system is leveraged) (feels like this
        should be in Design section, but at the same time this is pretty
        detailed that fits here)
    \item kernel helper and symbol bindings (dynamic linking scheme)
    \item kconfig-based conditional compilation
\end{itemize}

\subsection{Entry code generation}
\begin{itemize}
    \item LLVM pass
\end{itemize}

\subsection{Program load and attachment}
\begin{itemize}
    \item kernel loading code and attachment
    \item relocation fixups for maps and kernel symbols
    \item libiu
\end{itemize}

\subsection{Handle exceptional control flow}
\begin{itemize}
    \item kernel trampoline
\end{itemize}

\subsection{Stack overflow protection}
\begin{itemize}
    \item kernel vmapped, dedicated stack
    \item LLVM instrumentation
\end{itemize}

\subsection{program termination}
\begin{itemize}
    \item Need work
\end{itemize}
