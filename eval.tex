\section{Evaluation}

\begin{itemize}
    \item Macro-benchmark
        \begin{itemize}
            \item BMC
                \begin{itemize}
                    \item Expressiveness: how to evaluate?
                        \begin{itemize}
                            \item LOC: we have 33\% reduction
                            \item 1 rust program vs 7 eBPF programs
                            \item based on experience?
                            \item Rust is a safer language
                        \end{itemize}
                    \item Performance evaluation
                        \begin{itemize}
                            \item Check their paper to see if we can perform
                                the same experiment
                            \item We want a figure similar to Figure 6 in BMC
                                (except we don't need to evaluate on different
                                CPU configs)
                            \item x: Vanilla memcached, BMC, Rust
                            \item y: normalized throughput

                        \end{itemize}
                \end{itemize}
            \item Electrode
                \begin{itemize}
                    \item LOC reduction
                    \item Performance using their benchmark (similar to figure
                        5 and 7 in Electrode)
                    \item Dynamic allocation (ask the authors)
                \end{itemize}
            \item LSM implementation (if we have time / requires handle
                nesting)
            \item XRP
            \item FUSE in eBPF (from Hubertus)
        \end{itemize}
    \item Micro-benchmark
        \begin{itemize}
            \item Memory footprint (BPF vs Rust) given we have a larger binary
                \begin{itemize}
                    \item Number of prog in the same translation unit vs memory
                        usage for program allocation
                    \item This is because we are always statically linked to
                        the runtime crate on the translation unit basis
                    \item E.g. a bpf kern.c with 4 programs vs a rust main.rs
                        with 4 programs
                \end{itemize}
            \item Small expressiveness examples
                \begin{itemize}
                    \item Loop: strcmp
                    \item need more
                \end{itemize}
            \item Stack-check overhead
                \begin{itemize}
                    \item Use BMC?
                    \item Or create some other workload that are function call
                        intensive (since our instrumentation happens before
                        each function call)
                \end{itemize}
            \item Cleanup overhead on normal execution path (recording
                allocated kernel objects)
                \begin{itemize}
                    \item Similar to Stack-check
                \end{itemize}
            \item Startup overhead due to stack switching
                \begin{itemize}
                    \item A minimal program () to show the upper bound?
                    \item Plus a real world application (again BMC)?
                \end{itemize}
        \end{itemize}
\end{itemize}

\subsection{Macro-benchmarks}
\subsubsection{\projname{}-based BMC}
\jinghao{TODO: Preamable}

\para{Experiment setup} Our evaluation setup consists two machines, with one
    acting as the server and the other one acting as the client.
The server machine runs the \projname{} custom kernel on an AMD EPYC 7551P
    32-Core processor with 128 GB memory.
SMT and Turbo are turned off for the experiments.
The client machine runs a vanilla v6.7.8 Linux kernel on AMD Ryzen 9 7900X
    processor with 64 GB memory.
Both machines are equiped with Mellanox ConnectX-3 Pro 40GbE NICs and are
    connected back-to-back.