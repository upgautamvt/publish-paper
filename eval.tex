\section{Evaluation}

% \begin{itemize}
%     \item Macro-benchmark
%         \begin{itemize}
%             \item BMC
%                 \begin{itemize}
%                     \item Expressiveness: how to evaluate?
%                         \begin{itemize}
%                             \item LOC: we have 33\% reduction
%                             \item 1 rust program vs 7 eBPF programs
%                             \item based on experience?
%                             \item Rust is a safer language
%                         \end{itemize}
%                     \item Performance evaluation
%                         \begin{itemize}
%                             \item Check their paper to see if we can perform
%                                 the same experiment
%                             \item We want a figure similar to Figure 6 in BMC
%                                 (except we don't need to evaluate on different
%                                 CPU configs)
%                             \item x: Vanilla memcached, BMC, Rust
%                             \item y: normalized throughput
%
%                         \end{itemize}
%                 \end{itemize}
%             \item Electrode
%                 \begin{itemize}
%                     \item LOC reduction
%                     \item Performance using their benchmark (similar to figure
%                         5 and 7 in Electrode)
%                     \item Dynamic allocation (ask the authors)
%                 \end{itemize}
%             \item LSM implementation (if we have time / requires handle
%                 nesting)
%             \item XRP
%             \item FUSE in eBPF (from Hubertus)
%         \end{itemize}
%     \item Micro-benchmark
%         \begin{itemize}
%             \item Memory footprint (BPF vs Rust) given we have a larger binary
%                 \begin{itemize}
%                     \item Number of prog in the same translation unit vs memory
%                         usage for program allocation
%                     \item This is because we are always statically linked to
%                         the runtime crate on the translation unit basis
%                     \item E.g. a bpf kern.c with 4 programs vs a rust main.rs
%                         with 4 programs
%                 \end{itemize}
%             \item Small expressiveness examples
%                 \begin{itemize}
%                     \item Loop: strcmp
%                     \item need more
%                 \end{itemize}
%             \item Stack-check overhead
%                 \begin{itemize}
%                     \item Use BMC?
%                     \item Or create some other workload that are function call
%                         intensive (since our instrumentation happens before
%                         each function call)
%                 \end{itemize}
%             \item Cleanup overhead on normal execution path (recording
%                 allocated kernel objects)
%                 \begin{itemize}
%                     \item Similar to Stack-check
%                 \end{itemize}
%             \item Startup overhead due to stack switching
%                 \begin{itemize}
%                     \item A minimal program () to show the upper bound?
%                     \item Plus a real world application (again BMC)?
%                 \end{itemize}
%         \end{itemize}
% \end{itemize}

\subsection{Micro-benchmarks}
\subsubsection{Memory footprint (BPF vs Rust)}
\begin{itemize}
    \item need a figure: Number of progs in the same translation unit vs
        memory usage for program allocation
    \item x: Number of progs in a translation unit (empty prog with just a
        return) 1 to 8
    \item y: total number of bytes allocated for the loaded object (JIT-ed code
        for BPF, loaded and mapped pages for Rust)
\end{itemize}

\subsubsection{Startup overhead due to stack switching}
\begin{itemize}
    \item A table showing the runtime of an empty eBPF program and Rust program
        in terms of processor cycles (including the dispatcher function)
\end{itemize}

\subsubsection{Verifier/JIT-based optimization}
\begin{itemize}
    \item inline v.s. out-of-line map helpers
        \begin{itemize}
            \item table showing runtime of a map lookup operation in a eBPF and
                a Rust program
            \item the kernel will automatically inline lookup in BPF
            \item Rust version will not get inlined
            \item measurement can be in ns for now, but better if it could be
                in cycles
            \item maps to test: array, hash (htab-lru, xsk-map?)
        \end{itemize}
\end{itemize}

\subsubsection{Stack-check overhead}
\begin{itemize}
    \item table showing runtime of two programs using small recursion
    \item eBPF: use tail calls for recursion
    \item Rust: just a recursive function
    \item The recursion count should not be statically known to prevent llvm
        from optimizing out the recursion
\end{itemize}

\subsubsection{Cleanup overhead on normal execution path}
\begin{itemize}
    \item table showing runtime of two programs calling reference allocating
        helpers in a loop
    \item right now we only support spinlock (i.e. do a loop where can
        iteration lock and then unlock the spinlock)
\end{itemize}

\subsection{Macro-benchmarks}
We now demonstrate that \projname{} can be used to implement complicated,
    real-world kernel extension use cases with enhanced usability but without
%    losing much performance by implementing BMC and Electrode in \projname{}.
    losing much performance by implementing the BPF Memcached Cache in 
    \projname{}.

% \subsubsection{\projname{}-based BMC}
% \jinghao{TODO: Preamable}
% BMC implements in-kernel memcached cache -- how it works
% Compilcated program, original paper splits into 7 programs and uses tail calls
%

BMC~\cite{BMC} is an in-kernel cache for Memcached based on eBPF.
It stores recently queried key-value pairs in an eBPF map to accelerate the
    processing of GET requests to the Memcached server.
If a GET is hit in the cache, the queried value can be sent in a reply without
    going through the expensive network stack in the Linux kernel.
The map that servers as the cache is managed by eBPF programs, which implements
    the lookup and update logics.

On the aspect of implementation, BMC is a much more complicated program
    comparing to other common eBPF use cases.
In order to pass the verifier, the implementation has to be splitted into seven
    eBPF programs that tail-calls into each other to reduce verification
    complexity.
Similar problem also presents when processing the incoming packet data, where
    the authors had to bound data size to reduce verification complexity for
    loops.

We re-implement BMC in \projname{} (\projname{}-BMC) to demonstrate the
    enhanced usability without losing performance.
The resulting program is not a direct translation from the original eBPF
    version in C, rather, we implement the same high-level logic but with
    slight deviations from BMC where the enhance usability and expressiveness
    of Rust allows a simpler implementation.
\jinghao{should talk about some experience in implementing BMC, especially
    the parts we do differently.}

%\para{Experiment setup}
Our evaluation setup consists two machines, with one
    acting as the server and the other one acting as the client.
The server machine runs the \projname{} custom kernel based on Linux v5.15.0 on
    an AMD EPYC 7551P 32-Core processor with 112 GB memory.
SMT and Turbo are turned off for the experiments.
The client machine runs a vanilla v6.7.8 Linux kernel on AMD Ryzen 9 7900X
    processor with 64 GB memory.
Both machines are equiped with Mellanox ConnectX-3 Pro 40GbE NICs and are
    connected back-to-back.

% Key: 16 bytes, value: 32 bytes
% 100 million with Zipf 0.99
% 10 GB memcached, 2.5 GB BMC
% Preload all keys
% GET:SET 30:1

We use a workload with the same specs as in BMC.
Our dictionary contains 100 million Memcached keys following a Zipf
    distribution with 0.99 skewness.
All keys are 16 bytes in size and are paired with 32-byte values in the
    experiments.
The storage sizes of the Memcached and BMC are set to 10 GB and 2.5 GB,
    respectively.
According to the calculation in the original work, around 89 million items
    can be stored in Memcached itself and 6.3 million of them can fit in
    BMC.
Before experiments, the Memcached server is pre-loaded with all keys by
    sending TCP SET requests for each key in the dictionary from the client.
The client then sends requests to the server with a 30:1 ratio betweren UDP GET and
    TCP SET requests and measures the throughput.

We evaluate the throughput three setups: Memcached-SR (from BMC),
    Memcached-SR+BMC, and Memcached-SR+\projname{}-BMC.
The original open-sourced BMC code targets Linux kernel version 5.3.0 and we
    port it to the v5.15.0 \projname{} kernel for the evaluation.
For each setup, we vary the number of processor cores and the number of threads
    used by the Memcached server and pin each thread onto each core.
We also adjust the CPU affinity of IRQs associated with the NIC such that the
    network interrupts are processed on the same set of cores the Memcached
    server executes on.
