\section{Introduction}
Kernel extension is a powerful technique to improve performance and
    observability in operating systems.
eBPF is the current way of safely extending the Linux kernel.
The major use cases of eBPF today include networking, tracing and
    observability, and security.
eBPF has also attracted attention from the research community as
    a tool to speed up applications by reducing kernel crossings \cite{BMC,Electrode,bpfxrp}.

A main attractor of eBPF is its use of static verification to ensure extension safety.
This allows eBPF programs to be deployed in production to understand system performance,
    as well providing a programmable way to extend the kernel.
However, the use of a static verifier places unnecessary constraints on the expressiveness
    of extension programs and also makes them harder to develop and use.
eBPF projects contain a large number of workaround fixes to bypass the verifier and
    complexities that arise from the gap between the programmer, compiler, and verifier.

In this paper we present a novel extension system for the Linux kernel called \projname{}.
\projname{} uses a combination of the Rust compiler and dynamic mechanisms
    to ensure safety properties about kernel extensions.
At the same time, the use of Rust closes the gap between the programmer and verifier,
    as well as alleviates the expressiveness and usability challenges associated with
    the verifier.

We use our system to reimplement BPF Memcached Cache~\cite{BMC} that achieves
    comparable performance to the eBPF implementation, while avoiding the
    usability challenges of eBPF.

\begin{itemize}
    \item eBPF is the de-facto way of doing kernel extension in Linux
    \item Used in different domains
    \item
        \begin{itemize}
            \item networking, tracing, security
            \item also embraced by research community (BMC, XRP, Electrode)
        \end{itemize}
    \item Core value argument: verification for safety
    \item Problem: current static verification scheme (i.e. the verifier)
        places unnecessary constraints on expressiveness on extension programs
        \begin{itemize}
            \item BMC has a subsection discussing verification workarounds
            \item We had an unpleasant experience in porting BMC (though this
                is more like the new compiler does not play well with the
                verifier for some old code)
            \item Overhead argument in BMC (sending data through map vs
                function arguments)
            \item Certain program constructs are not possible
            \item Some more evidence/example needed
        \end{itemize}
    \item Another point to include: for certain safety properties static
        verification is fundamentally limited even in current eBPF
        \begin{itemize}
            \item From Roop's experiment: there is not a way for verifier to
                figure out statically how much stack will be used when bpf2bpf
                calls and tail calls are mixed due to the indirect nature of
                tail calls. If the stack is 8k (e.g. on 32-bit platforms) the
                verifier cannot protect the stack.
            \item Related to our argument on runtime mechanism
        \end{itemize}
    \item Our solution: we should use a more expressive language for kernel
        extensions and move away from the verifier. The language should:
        \begin{itemize}
            \item be more expressive (e.g. Turing complete)
            \item support equivalent safety properties as the verifier (the
                hotos table)
                \begin{itemize}
                    \item memory safety
                    \item control-flow safety
                    \item type safety
                    \item safe resource management
                    \item program termination
                    \item kernel stack overflow protection
                \end{itemize}
            \item Rust
                \begin{itemize}
                    \item a widely used high-level programming language, also
                        embraced by Linux (Rust for Linux)
                    \item happens to have most of these properties out-of-box,
                        therefore we choose to build upon Rust
                \end{itemize}
        \end{itemize}
\end{itemize}
